\documentclass{article}
\usepackage{amsmath}
\usepackage{amsthm}
\usepackage{amssymb}
\usepackage{hyperref}
\RequirePackage[left=3.9cm,right=3.9cm, top=2cm,bottom=2cm,bindingoffset=0cm]{geometry}
\title{Algebra Definitions}
\author{eeleexx}

\newtheorem*{customdefinition}{Definition}

\begin{document}

\maketitle

\section*{Preface}
The definitions included in this document are taken from the lecture notes by Dima Trushin. The original notes can be found in the GitHub repository at

\url{https://github.com/DimaTrushin/Algebra-DSBA}.

The repository is open to suggestions, and anybody is welcome to contribute by making pull requests at \url{https://github.com/eeleexx/AlgebraOralTest}. Don't hesitate to report any inconsistencies or errors in definitions, as well as if some of them are missing or blatantly incorrect.

\subsection*{A binary operation. Definition 4}
\begin{customdefinition}
Suppose \(X\) is a set. A binary operation is a map \(\circ : X \times X \rightarrow X\) by the rule \((x, y) \mapsto x \circ y\) for \(x, y \in X\).

In this case, the notation \(\circ\) is the name of the operation. Simply speaking, the operation is a rule that takes two elements of the set \(X\) and produces a new element called \(x \circ y\) of the same set \(X\). This element \(x \circ y\) is usually called the product of \(x\) and \(y\).
\end{customdefinition}

\subsection*{Associative operation. Definition 7}
\begin{customdefinition}
An operation \(\circ : X \times X \rightarrow X\) is called associative if for every element \(x, y, z \in X\) we have \((x \circ y) \circ z = x \circ (y \circ z)\).
\end{customdefinition}

\subsection*{A neutral element. Definition 9}
\begin{customdefinition}
Let \(\circ : X \times X \rightarrow X\) be an operation on \(X\). An element \(e \in X\) is called neutral (or identity element) if for every element \(x \in X\) we have \(x \circ e = x\) and \(e \circ x = x\).
\end{customdefinition}

\subsection*{An inverse element in case of a binary operation. Definition 12}
\begin{customdefinition}
Let \(\circ : X \times X \rightarrow X\) be an operation such that there is a neutral element \(e \in X\). An element \(y \in X\) is called inverse to an element \(x \in X\) if \(x \circ y = e\) and \(y \circ x = e\).
\end{customdefinition}

\subsection*{A group. Definition 17}
\begin{customdefinition}
\textbf{Definition of a group.}
\begin{itemize}
    \item \textbf{Data:}
        \begin{itemize}
            \item A set \(G\).
            \item An operation \(\circ : G \times G \rightarrow G\).
        \end{itemize}
    \item \textbf{Axioms:}
        \begin{itemize}
            \item The operation \(\circ\) is associative.
            \item The operation \(\circ\) has a neutral element.
            \item Every element \(x \in G\) has an inverse.
        \end{itemize}
\end{itemize}
In this case, we say that the pair \((G, \circ)\) is a group. In order to simplify the notation, we usually say simply that \(G\) is a group assuming that the operation in use is clear. If in addition we have:
\begin{itemize}
    \item The operation \(\circ\) is commutative.
\end{itemize}
Then the group \(G\) is called abelian or simply commutative.
\end{customdefinition}

\subsection*{An abelian group. Definition 17}
\begin{customdefinition}
Let \( G \) be a group with operation \( \circ \). If the operation \( \circ \) is commutative, that is, for every \( x, y \in G \) we have \( x \circ y = y \circ x \), then \( G \) is called an abelian group.
\end{customdefinition}

\subsection*{The group \( \mathbb{Z}_n \). Example 18 item 4}
\begin{customdefinition}
Let \( n \) be any positive integer. The set \( \mathbb{Z}_n = \{0, 1, \ldots, n-1\} \) with addition modulo \( n \) is an abelian group. The operation on \( \mathbb{Z}_n \) will be simply denoted by \( + \).
\end{customdefinition}

\subsection*{The group \( \mathbb{Z}^*_n \). Example 18 item 5}
\begin{customdefinition}
Let \( n \) be any positive integer. The set \( \mathbb{Z}^*_n = \{ m \in \mathbb{Z}_n \mid \gcd(m, n) = 1 \} \) with multiplication modulo \( n \) is an abelian group. The operation on \( \mathbb{Z}^*_n \) will be simply denoted by \( \cdot \).
\end{customdefinition}

\subsection*{A subgroup. Definition 19}
\begin{customdefinition}
Let \( G \) be a group with operation \( \circ \). A subset \( H \subseteq G \) is called a subgroup if \( H \) itself forms a group with the inherited operation \( \circ \) from \( G \). This means that \( H \) must satisfy the group axioms: associativity, identity element, inverse elements, and closure under the operation \( \circ \).
\end{customdefinition}

\subsection*{A cyclic subgroup. Definition 22}
\begin{customdefinition}
Let \( G \) be a group and \( g \in G \) be an arbitrary element. The cyclic subgroup generated by \( g \) is the set \( \langle g \rangle = \{ g^n \mid n \in \mathbb{Z} \} \), where \( g^n \) denotes the \( n \)-th power of \( g \) (using the group operation \( \circ \) repeatedly). In multiplicative notation, \( g^n = g \circ g \circ \ldots \circ g \) ( \( n \) times), and in additive notation, \( ng = g + g + \ldots + g \) ( \( n \) times).
\end{customdefinition}

\subsection*{The order of an element of a group. Definition 24}
\begin{customdefinition}
Let \( G \) be a group and \( g \in G \) be an arbitrary element. Then there are two options:
\begin{itemize}
    \item If \( \operatorname{ord}(g) = \infty \), then the elements \( g^n \) and \( g^m \) are different whenever \( n, m \in \mathbb{Z} \) are different.
    \item If \( \operatorname{ord}(g) = n < \infty \), then elements \( 1, g, g^2, \ldots, g^{n-1} \) are different. In this case, the powers are repeated in cycles, that is in the series
\[
\ldots, g^{-2}, g^{-1}, 1, g, g^2, \ldots, g^{n-1}, g^n, g^{n+1}, \ldots, g^{2n-1}, g^{2n}, \ldots
\]
are the same elements as \( 1, g, \ldots, g^{n-1} \) for any \( k \in \mathbb{Z} \). In particular, \( \langle g \rangle = \{1, g, \ldots g^{n-1}\} \).
\end{itemize}
\end{customdefinition}

\subsection*{A coset in a group. Definition 29}
\begin{customdefinition}
Let \( G \) be a group, \( H \subseteq G \) a subgroup, and \( g \in G \) an arbitrary element. Then the set \( gH = \{gh \mid h \in H\} \) is called the left coset of \( H \) with respect to \( g \). In a similar way, we define right cosets. The set \( Hg = \{hg \mid h \in H\} \) is called the right coset of \( H \) with respect to \( g \).
\end{customdefinition}



\subsection*{A normal subgroup. Definition 32}
\begin{customdefinition}
Let \( G \) be a group and \( H \) its subgroup. The subgroup \( H \) is normal if its left and right cosets are the same, that is, \( gH = Hg \) whenever \( g \in G \).
\end{customdefinition}

\subsection*{The index of a subgroup. Definition 38}
\begin{customdefinition}
Let \( G \) be a finite group and \( H \subseteq G \) a subgroup. Then the number of the left cosets of \( H \) is called the index of \( H \) and is denoted by \( (G : H) \). This number also coincides with the number of right cosets of \( H \).
\end{customdefinition}

\subsection*{A homomorphism of groups. Definition 40}
\begin{customdefinition}
Let \( G \) and \( H \) be groups. A homomorphism \( \varphi : G \rightarrow H \) is a map such that \( \varphi(g_1 g_2) = \varphi(g_1) \varphi(g_2) \) for any \( g_1, g_2 \in G \). In this case, \( \varphi \) is called a homomorphism from \( G \) to \( H \).
\end{customdefinition}

\subsection*{An isomorphism of groups. Definition 44}
\begin{customdefinition}
Let \( G \) and \( H \) be groups. We define an isomorphism \( \varphi : G \rightarrow H \).

\begin{itemize}
    \item \textbf{Data:} A homomorphism \( \varphi : G \rightarrow H \).
    \item \textbf{Axiom:} \( \varphi \) is bijective.
\end{itemize}

In this case, \( \varphi \) is called an isomorphism between \( G \) and \( H \). If there is an isomorphism between \( G \) and \( H \), the groups \( G \) and \( H \) are called isomorphic.
\end{customdefinition}

\subsection*{The kernel of a homomorphism of groups. Definition 46 item 1}
\begin{customdefinition}
Let \( \varphi : G \rightarrow H \) be a homomorphism of groups. The kernel of \( \varphi \) is \( \ker \varphi = \{ g \in G \mid \varphi(g) = 1 \} \subseteq G \).
\end{customdefinition}

\subsection*{The image of a homomorphism of groups. Definition 46 item 2}
\begin{customdefinition}
Let \( \varphi : G \rightarrow H \) be a homomorphism of groups. The image of \( \varphi \) is \( \operatorname{Im} \varphi = \{ \varphi(g) \mid g \in G \} = \varphi(G) \subseteq H \).
\end{customdefinition}

\subsection*{A product of groups. Definition 48}
\begin{customdefinition}
Let \( G \) and \( H \) be groups. We define a new group \( G \times H \) as follows:
\begin{itemize}
    \item As a set, it is the product of the underlying sets of the groups: \( G \times H = \{ (g, h) \mid g \in G, h \in H \} \).
    \item The operation \( \cdot : (G \times H) \times (G \times H) \rightarrow G \times H \) is given by the rule \((g_1, h_1)(g_2, h_2) = (g_1 g_2, h_1 h_2)\), for \( g_1, g_2 \in G \) and \( h_1, h_2 \in H \).
\end{itemize}
The group \( G \times H \) is called the product of the groups \( G \) and \( H \).
\end{customdefinition}

\subsection*{A ring. Definition 60}
\begin{customdefinition}
A ring is a set \( R \) equipped with two binary operations \( + \) and \( \cdot \) (addition and multiplication) such that:
\begin{itemize}
    \item \( (R, +) \) is an abelian group.
    \item \( \cdot \) is associative.
    \item \( \cdot \) is distributive over \( + \).
\end{itemize}
\end{customdefinition}

\subsection*{A field. Definition 60}
\begin{customdefinition}
A ring is called a field if it satisfies the following conditions:
\begin{itemize}
    \item Every non-zero element is invertible with respect to multiplication: for every \( a \in R \setminus \{0\} \), there exists an element \( b \in R \) such that \( ab = ba = 1 \).
    \item \( 1 \neq 0 \).
\end{itemize}
In this case, the inverse element for \( a \) is denoted by \( a^{-1} \).
\end{customdefinition}

\subsection*{The ring \( \mathbb{Z}_n \). Example 61 item 5}
\begin{customdefinition}
The set of remainders modulo natural number \( n \) with the usual addition and multiplication modulo \( n \), that is \( (\mathbb{Z}_n, +, \cdot) \), is a commutative ring.
\end{customdefinition}

\subsection*{A subring. Definition 63}
\begin{customdefinition}
Let \( R \) be a ring. We are going to define a subring \( T \subseteq R \).

\begin{itemize}
    \item \textbf{Data:} 
        \begin{itemize}
            \item A subset \( T \subseteq R \).
        \end{itemize}
    \item \textbf{Axioms:}
        \begin{itemize}
            \item \( (T, +) \subseteq (R, +) \) is a subgroup.
            \item \( T \) is closed under multiplication.
            \item \( T \) contains 1.
        \end{itemize}
\end{itemize}
\end{customdefinition}

\subsection*{Invertible elements, zero divisors, nilpotent and idempotent elements. Definition 65}
\begin{customdefinition}
Let \( R \) be a ring and \( x \in R \) be an element of \( R \).

\begin{itemize}
    \item The element \( x \) is called invertible if there exists \( y \in R \) such that \( xy = yx = 1 \). In this case \( y \) is denoted by \( x^{-1} \). The set of all invertible elements of \( R \) is denoted by \( R^* \).
    
    \item The element \( x \) is called left zero divisor if there exists a nonzero \( y \in R \) such that \( xy = 0 \). Similarly, \( x \) is called right zero divisor if there exists a nonzero \( y \in R \) such that \( yx = 0 \). The sets of left and right zero divisors will be denoted by \( D_l(R) \) and \( D_r(R) \), respectively. The set \( D(R) = D_l(R) \cup D_r(R) \) is the set of all zero divisors of \( R \).
    
    \item The element \( x \) is called nilpotent if \( x^n = 0 \) for some \( n \in \mathbb{N} \). The set of all nilpotent elements is denoted by \( \operatorname{nil}(R) \).
    
    \item The element \( x \) is called idempotent if \( x^2 = x \). The set of all idempotents of \( R \) is denoted by \( E(R) \).
\end{itemize}
\end{customdefinition}

\subsection*{An ideal. Definition 67}
\begin{customdefinition}
Suppose that $(R,+, \cdot)$ is a ring. An ideal $I$ in the ring $R$ is defined as follows:
\begin{itemize}
    \item \textbf{Data:}
        \begin{itemize}
            \item A subset $I \subseteq R$.
        \end{itemize}
    \item \textbf{Axioms:}
        \begin{itemize}
            \item $(I,+) \subseteq (R,+)$ is a subgroup.
            \item For any $r \in R$ we have
                \[
                r I = \{rx\mid x\in I\} \subseteq I\quad\text{and}\quad Ir = \{xr\mid x\in I\}\subseteq I
                \]
        \end{itemize}
\end{itemize}
In this case, we say that $I$ is an ideal of $R$. The subsets $0$ and $R$ are always ideals and are called the trivial ideals of $R$.
\end{customdefinition}

\subsection*{A homomorphism of rings. Definition 70}
\begin{customdefinition}
Let $(R,+, \cdot)$ and $(S,+, \cdot)$ be rings. A homomorphism $\phi : R \rightarrow S$ is defined as follows:
\begin{itemize}
    \item \textbf{Data:}
        \begin{itemize}
            \item A map $\phi : R \rightarrow S$.
        \end{itemize}
    \item \textbf{Axioms:}
        \begin{itemize}
            \item $\phi(a+b) = \phi(a) + \phi(b)$ for all $a, b \in R$.
            \item $\phi(ab) = \phi(a) \phi(b)$ for all $a, b \in R$.
            \item $\phi(1) = 1$.
        \end{itemize}
\end{itemize}
In this case, we say that $\phi$ is a homomorphism from $R$ to $S$. If in addition $\phi$ is bijective, then $\phi$ is called an isomorphism, and $R$ and $S$ are called isomorphic.
\end{customdefinition}

\subsection*{The kernel of a ring homomorphism. Definition 74}
\begin{customdefinition}
Let $\phi : R \rightarrow S$ be a homomorphism of rings. Then:
\begin{itemize}
    \item The kernel of $\phi$ is $\ker \phi = \{r \in R \mid \phi(r) = 0\} \subseteq R$.
    \item The image of $\phi$ is $\operatorname{Im} \phi = \{\phi(r) \mid r \in R\} = \phi(R) \subseteq S$.
\end{itemize}
\end{customdefinition}

\subsection*{A greatest common divisor of two polynomials. Definition 81}
\begin{customdefinition}
Let \( F \) be a field and \( f, g \in F[x] \) be some polynomials. A polynomial \( d \in F[x] \) is called a greatest common divisor of \( f \) and \( g \) if:
\begin{itemize}
    \item \( d \) divides both \( f \) and \( g \).
    \item if \( h \) divides both \( f \) and \( g \), then \( h \) divides \( d \).
    \item \( d \) is monic.
\end{itemize}
\end{customdefinition}

\subsection*{An irreducible polynomial in one variable. Definition 86}
\begin{customdefinition}
A polynomial \( f \in F[x] \setminus F \) is irreducible if for any \( g, h \in F[x] \) such that \( f = gh \), either \( g \) or \( h \) is a nonzero constant.
\end{customdefinition}

\subsection*{The ring of polynomial remainders}
Let \( F \) be a field and \( f \in F[x] \) be any polynomial. I am going to define the ring \( F[x]/(f) \). First, I need to specify a set, then two operations: addition and multiplication, and finally, I should check all the axioms. If \( f = 0 \), we define \( F[x]/(f) \) to be the polynomial ring itself \( F[x] \). The interesting case is when \( f \neq 0 \):
\begin{itemize}
    \item \( F[x]/(f) = \{g \in F[x] \mid \deg g < \deg f \} \) the set of remainders with respect to \( f \).
    \item \( +: F[x]/(f) \times F[x]/(f) \rightarrow F[x]/(f) \) is the usual addition of polynomials.
    \item \( \cdot : F[x]/(f) \times F[x]/(f) \rightarrow F[x]/(f) \) is the multiplication modulo \( f \), namely: for every \( g, h \in F[x]/(f) \), we define \( gh \mod f \). The latter means, we divide \( gh \) by \( f \) with remainder and get \( gh = qf + r \). Then the product of \( g \) and \( h \) is \( r \).
\end{itemize}

\subsection*{The characteristic of a field. Definition 93}
\begin{customdefinition}
Let \( F \) be a field. The characteristic of \( F \) is the minimal positive integer \( p \) such that
\[
\underbrace{1 + 1 + \cdots + 1}_{p \text{ times}} = 0.
\]
If there is no such \( p \), the characteristic is said to be zero. The characteristic of \( F \) is denoted by \( \operatorname{char} F \).

To introduce a convenient notation, if we add an element \( x \in F \) \( n \) times, where \( n \in \mathbb{N} \), we may denote the sum as follows:
\[
nx = \underbrace{x + x + \cdots + x}_{n \text{ times}}.
\]

In particular, the characteristic of \( F \) is the smallest positive integer \( p \) such that \( p \cdot 1 = 0 \).
\end{customdefinition}


\subsection*{An extension by a root for fields. Section 7.2}
nah i didnt find it

\subsection*{A lexicographical order on monomials. Definition 108}
\begin{customdefinition}
We want to define a lexicographical order on monomials.
\begin{enumerate}
    \item We need to fix an ordering on the variables \(x_1, \ldots, x_n\). For example \(x_1 > x_2 > \ldots > x_n\). However, we can take any permutation of the variables.
    \item Suppose we fixed the ordering \(x_1 > \ldots > x_n\) on the variables. Now, we are ready to define the corresponding lexicographical order \(\text{Lex}(x_1, \ldots, x_n)\) on the monomials.
\end{enumerate}
Let \(m = x_1^{k_1} \ldots x_n^{k_n}\) and \(m' = x_1^{k'_1} \ldots x_n^{k'_n}\) be two monomials. Then we compare \(k_1\) and \(k'_1\). If \(k_1 > k'_1\), then \(m > m'\). If \(k_1 < k'_1\), then \(m < m'\). If \(k_1 = k'_1\), then we compare \(k_2\) and \(k'_2\) and repeat the algorithm above. In particular, \(m > m'\) if and only if there exists \(1 \leq j \leq n\) such that \(k_1 = k'_1, \ldots, k_{j-1} = k'_{j-1}\) and \(k_j > k'_j\).
\end{customdefinition}

\subsection*{The leading term of a polynomial. Definition 113}
\begin{customdefinition}
Suppose \(F\) is a field and we fix some lexicographical order on the monomials in \(n\) variables. As before, each polynomial \(f \in F[x_1, \ldots, x_n]\) can be written as
\[
f = c_1 m_1 + c_2 m_2 + \ldots + c_k m_k,
\]
\[
\quad c_i \in F, \, m_i \text{ are monomials such that } m_1 > m_2 > \ldots > m_k.
\]
We denote its \(i\)-th largest monomial \(m_i\) by \(M_i(f)\). In particular, \(M_1(f)\) is the leading monomial of \(f\), \(M_2(f)\) is the next largest monomial of \(f\) etc. The \(i\)-th largest monomial need not exist if \(f\) contains less than \(i\) monomials.
\end{customdefinition}

\pagebreak

\subsection*{An elementary reduction of a polynomial with respect to another one. Definition 114}
\begin{customdefinition}
Suppose \( g \in F[x_1, \ldots, x_n] \) is a nonzero polynomial and \( f \in F[x_1, \ldots, x_n] \) is any polynomial. Assume that
\[
f = c_1 m_1 + \ldots + c_i m_i + \ldots + c_k m_k, 
\]
\[
    \quad c_i \in F, \, m_i \text{ are monomials such that } m_1 > m_2 > \ldots > m_k
\]
and
\[
g = C(g)M(g) + g_0 = T(g) + g_0.
\]
We take \( m \) to be \( m_i \), that is a monomial in \( f \), and assume that \( m \) is divisible by the leading monomial of \( g \), that is \( m = tM(g) \). We define an elementary reduction of \( f \) with respect to \( g \) as
\[
f \xrightarrow{g} f' = f - \frac{c_i}{C(g)} t g.
\]
The polynomial \( f' \) is the result of the elementary reduction.

In short, the elementary reduction works as follows: we find a monomial \( m_i \) of \( f \) divisible by \( M(g) \) and replace it by the tail of \( g \) multiplied by \( -\frac{c_i m_i}{T(g)} \).
\end{customdefinition}

\subsection*{A reduction and a remainder of a polynomial with respect to a set of nonzero polynomials. Definition 116}
\begin{customdefinition}
Suppose \( G \subseteq F[x_1, \ldots, x_n] \setminus \{0\} \) is a set of polynomials and \( f, f' \in F[x_1, \ldots, x_n] \) are any polynomials. We say that \( f \) is reducible to \( f' \) with respect to \( G \) if there is a finite sequence of elementary reductions as below:
\[
f \xrightarrow{g_1} f_1 \xrightarrow{g_2} f_2 \xrightarrow{g_3} \ldots \xrightarrow{g_k} f_k = f' \quad \text{where } g_i \in G.
\]
In this case, we will write 
\[
f \stackrel{G}{\rightsquigarrow} f'.
\]
If the polynomial \( f' \) is not reducible by any \( g \in G \), we say that \( f' \) is a remainder of \( f \) with respect to \( G \).
\end{customdefinition}



\subsection*{Gröbner basis. Definition 118}
\begin{customdefinition}
Suppose \( F \) is a field, \( G \subseteq F[x_1, \ldots, x_n] \setminus \{0\} \), and we fix a lexicographical order on the monomials. We say that \( G \) is a Gröbner basis if for every \( f \in F[x_1, \ldots, x_n] \) all its remainders are the same.
\end{customdefinition}

\subsection*{S-polynomial. Definition 123}
\begin{customdefinition}
Suppose \( F \) is a field, \( f_1, f_2 \in F[x_1, \ldots, x_n] \) are some nonzero polynomials, and we are given a lexicographical order on monomials. Assume that 
\[
f_1 = c_1 m_1 + f'_1,
\]
where \( c_1 m_1 \) is the leading term, and
\[
f_2 = c_2 m_2 + f'_2,
\]
where \( c_2 m_2 \) is the leading term. Let \( m \) be the least common multiple of \( m_1 \) and \( m_2 \), then \( m = m_1 t_1 = m_2 t_2 \). Then, the polynomial
\[
S_{f_1, f_2} = c_2 t_1 f_1 - c_1 t_2 f_2 = c_2 t_1 f'_1 - c_1 t_2 f'_2
\]
is called the S-polynomial of \( f_1 \) and \( f_2 \).
\end{customdefinition}

\subsection*{A finitely generated ideal. Definition 129}
\begin{customdefinition}
Suppose \( F \) is a field and we are given a finite set of polynomials \( g_1, \ldots, g_k \in F[x_1, \ldots, x_n] \). Then the set
\[
(g_1, \ldots, g_k) = \{ g_1 h_1 + \ldots + g_k h_k \mid h_1, \ldots, h_k \in F[x_1, \ldots, x_n] \}
\]
is an ideal of \( F[x_1, \ldots, x_n] \) and is called the ideal generated by \( g_1, \ldots, g_k \). If we take \( G = \{ g_1, \ldots, g_k \} \), then the ideal \( (g_1, \ldots, g_k) \) is also denoted by \( (G) \) for short.
\end{customdefinition}


\end{document}
